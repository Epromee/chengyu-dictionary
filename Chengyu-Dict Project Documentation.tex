\documentclass[11pt]{article} %, twocolumn if need to be
\usepackage{geometry}
\geometry{a4paper}
\usepackage[parfill]{parskip}    		% Activate to begin paragraphs with an empty line rather than an indent
\usepackage{graphicx}					
\usepackage{times} % Times font
\usepackage{url,hyperref}	
\usepackage{amssymb}
\usepackage{amsmath}
\usepackage{xeCJK}
%\setCJKmainfont{SimSun}

\title{Chengyu Dictionary Documentation} % Project Documentation Title
\author{Xuefeng Luo \and Jingwen Li}
\date{}	% no date
\begin{document}
\maketitle

\section{Introduction}
\indent The Internet has promoted globalisation and along with it, the urge (or necessity) to be multilingual. Following this trend, Chinese learners worldwide have also increased in number. As an inseparable part of the Chinese language, \textit{Chengyu}, or the so-called "four-character idioms", is on the other hand difficult to master. One reason behind this is that there are simply so many of them and it is hard for a learner to learn them in chunks. Most of the \textit{Chengyu} are learned as they are encountered, which surely satisfies most learners practically, but not the hungry ones. To help the intermediate/advanced Chinese learners, we present this Chengyu Dictionary as an entry-point to tame these culturally rich beasts.\\
\\
\indent This paper is presented in the following structure:\\
Section 2 talks about the existing tools and a paper on translating Chengyu into English. Section 3 lists the project dependencies. Section 4 is about data processing, including preprocessing the corpora, translation difficulties and other lexicographic decisions made. Section 5 deals with the search mechanism with technical details. Then, in section 6, we present the user interface. Section 7 recorded the division of labor and section 8 is our discussion of this tool and vision for future improvements.

\section{Related Works}
The following are existing tools that share similar functions with our web application:
\begin{itemize}
\item Chengyu Dictionary provided by \textit{Chinese-Tools.com}\\
\textit{Chengyu Dictionary} is an online dictionary that provide pinyin, explanations in English (only for some entries) and detailed explanations in Chinese, including meaning, context, example, synonyms, antonyms and even grammatical information. A user may search by pinyin or Chinese character input, but not English. The results are displayed in a list of links which lead to the entries' own page, where the above information are presented.\\
\\
Available at: https://www.chinese-tools.com/chinese/chengyu/dictionary\\
\item Google-Translate for translating Chengyu\\
One may also input a Chengyu into Google Translate, but the translation is most of the time not accurate enough and is without context. No additional information is provided except the translation. \\
\end{itemize}
There is also a paper on translation strategies: \textit{A Study of Idiom Translation Strategies between English and Chinese}\\
This paper discussed the different strategies that are commonly used when translating idiomatic phrases, Chengyu in particular, into English. The four main strategies are: literal translation, free translation, abridged translation and borrowing translation. It also outlined the principles to live by when translating idioms.


\section{Project Dependencies}
A list of project dependencies used in this project (managed by Maven):
\begin{itemize}
\item Google Web Toolkit and Mojo's Maven Plugin for GWT  (gwt-maven-plugin, version 2.8.1)
\item GwtBootstrap3 and GwtBootstrap3-extras (gwtbootstrap3 and gwtbootstrap3-extras, version 0.9.4) for button group, button dropdown and multiple select.
\end{itemize}

\section{Method}
\subsection{Pre-processing Steps}
\indent Talk about selection of Corpora, sort by frequency etc.\\
\subsubsection{Chengyu Data}
Available at: https://github.com/by-syk/chinese-idiom-db\\
\subsubsection{Corpus: OpenSubtitles}
\indent OpenSubtitles: blah\\
\subsubsection{Determine Frequency of Chengyu}
\indent Cross-reference with data from Github, order by frequency, first 150 entries are translated
\subsubsection{Convert to JSON}
\subsubsection{Translation}
\subsubsection{MySQL Database}

\subsection{Linguistic Problems}

\indent blah.Talk about Chengyu and its status in Chinese. Talk about how the choice of corpora might have affected the frequency observed. Talk about translation difficulties.\\
Chengyu and Chinese.\\
Written language and spoken language differ, film language is more so. Film subtitles might subject to the limit of space and time, to use them in a somewhat unnatural way. One improvement to be made could be to incorporate different kinds of corpora. But then there is also a problem with time. Literature might be old and language might be archaic. In this sense, since film are rather recent, could be a good choice.\\

\subsection{Lexicographic Decisions}

\indent How to organise the entries. What information to include. \\

\section{Search Mechanism}

\indent blah\\

\section{User Interface}

\indent blah. insert a screenshot. blah.\\

\section{Division of Labor}

\indent blah\\

\section{Discussion and Future Work}
\indent What we have accomplished with regard to the initial plan we set out. What didn't. I guess one thing is that we did not do the origin translation. Pinyin search.

\indent What could be improved given more time invested on the project. 

\section{What Heitu wrote}
\subsection{Search}
\indent Initially, we have three search methods. There are three sections over the input box. Users can search by Chinese, Pinyin and English. Currently, searching by Chinese is perfect, but Pinyin and English are not supported fully. If a user searches by Pinyin, he could only search by one character. For example, ideally, "世外桃源" should be searched by "shi wai tao yuan" or a partial pattern. However, it can be searched only by "shi" or "wai" or "tao" or "yuan". Moreover, the user could also search a chengdu while indicating its tags or domains. For example, when a user want to search "无" with only negative attitude, he can select "negative" tag under filter section. Then, all chengyus whcih contain charactor "无" will display in the display section. 

\subsection{User Interface}
\indent In total, we have three sections for our main page. First of all, our title section locate in the top middle. Below that, it is our search section where there are one input box, three navigate button, a dropdown selection box to select tags and two major button to either search a chengdu or download xml file. After that, we have one more section to display the search results. The results are displayed in a table where the columns contains Chinese, pinyin, figurative English translation, extended English meaning, example, example translation and frequency. Domains as known as tags, designed as badges, are attached at the beginning of each row.

\subsection{Search Mechanism}
\indent We split our search function into two parts. The initial search and tag-filter. In our initial search, we first decide which mode the user chose and pass target and mode variables to our search logic class. In this class, we already made SQL statement templates and insert variables into those statement and request database to search. After database returning back, we retrieve the data and pasre the data into an arraylist of Entries. Then, we used the pre-made function to filter the results and sent them to the front.

\subsection{Database}
\indent We are using MySQL which is the world most common used and free relational database and it is easy to used and maintain. To design our data tables, we follow first, second and third normalization where we eliminate all repeat records, partial dependency and transfer dependency. Thus, we have chengyu table and tags table and one more assiociate table which stores the relationships of chengyu and tags where we have IDs for both chengyu and tags as primary keys and foreign keys.

\subsection{Frequency}
\indent In order to allow Chinese Chengyu learner to learn which chengyu are mostly common used and which chengyus are not. We count each chengyu within a movie subtitile database. Though there are limited resource of chengyu usage, we managed to draw a general idea of how chengyus are used. To count the appearances of each chengyu, we create a separate python program located in the data folder.

\subsection{How to install}
To install our application, please follow the listed instructions:

\begin{enumerate}
  \item First, you need an environment where GWT can be deployed. At this moment, we are using Eclipse plus a GWT plugin where we retrieved it from Eclipse market place.
  \item Then, MySQL is needed in order to import our data.
  \item MySQL Workbench is optional but it helped comparing to MySQL command line tools.
  \item To import our data, you need to run data\_structure.sql file where our data structures and relations were stored.
  \item Then, you should first import Chengyu data from chengyu\_data.json file and tags data from tags.csv file, since there are foreign key dependencies.
  \item Then, you can import chengyu tags relations from chengyu\_tag.csv file.
  \item Following that, you need to go to DictionaryServiceImpl.java \\(located in chengyu.dict\textbackslash src\textbackslash main\textbackslash java\textbackslash com\textbackslash colewe\textbackslash ws1819\textbackslash server)\\ where database connection information was hard-coded in. Variables DB\_URL, USER and PASS need to be modified in order to make our program connect to the database.
  \item Please run the application through Eclipse.
  \item Open the page \text{http://localhost:8888/chengyudict} via a browser. We have tested and passed with Safari, Google Chrome and Firefox.
  \item Enjoy it!
\end{enumerate}

\subsection{ref list}
\indent open-subtitle database

\bibliographystyle{abbrv}
\bibliography{refs}

\end{document}  